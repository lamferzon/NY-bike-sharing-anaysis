\documentclass[a4paper, 12pt, one column, aas_macros]{article}

% Language and font encodings. This says how to do hyphenation on end of lines.
\usepackage[english]{babel}
\usepackage[utf8x]{inputenc}
\usepackage[T1]{fontenc}

% Sets page size and margins. You can edit this to your liking
\usepackage[top=1.3cm, bottom=2.0cm, outer=2.5cm, inner=2.5cm, heightrounded,
marginparwidth=1.5cm, marginparsep=0.4cm, margin=2cm]{geometry}

% Useful packages
\usepackage{graphicx} %allows you to use jpg or png images. PDF is still recommended
\usepackage[colorlinks=False]{hyperref} % add links inside PDF files
\usepackage{amsmath}  % Math fonts
\usepackage{amsfonts} %
\usepackage{amssymb}  %

% Citation package
\usepackage[authoryear]{natbib}
\bibliographystyle{abbrvnat}
\setcitestyle{authoryear,open={(},close={)}}

\usepackage{pdfpages}
\usepackage{graphicx}
\usepackage{subfigure}
\usepackage{array}
\usepackage{siunitx}
\usepackage{booktabs}
\usepackage{import}

\newcommand\todo[1]{\textcolor{red}{#1}}

\title{Statistical analysis of Jersey City bike sharing data using spatio-temporal models}
\author{Alessandro CHAAR \and Lorenzo LEONI \and Nicola ZAMBELLI}
\date{%
	University of Bergamo, Department of Management, Information and Production Engineering\\[2ex]%
	\today
}

\begin{document}
	\maketitle
	
	\begin{abstract}
		Is it possible to predict how many bikes to place at a pickup station to optimize the bike sharing service? Starting from \num{2020} Jersey City (NYC) bike staring data, the aim of this study is to find an answer to this question estimating three different spatio-temporal models. The historical weather data (and not only) will try to explain the daily (and hourly) number of pickups at a station in order to help the service provider in its planning. \\\\
		\textbf{Keywords}: bike sharing, DCM, HDGM, f-HDGM, D-STEAM.
	\end{abstract}
	
	\import{Sections/}{Dataset description.tex}
	\import{Sections/}{Scientific questions.tex}
	\import{Sections/}{Methodology.tex}
	\import{Sections/}{Data analysis.tex}
	\import{Sections/}{Results comparison.tex}
	\import{Sections/}{Conclusions.tex}
	
\end{document}